\section{Introduction}
%<Building Research Partnerships >: Method for Building a Meaningful Disability-Related Research Agenda
%.... Disabilities via their content
%This method bridges close readings from the humanities with technology design related to disability.
In this paper, we propose a novel 5-step method for building empathy with people with disabilities (PwD) that leverages existing content created by these populations. Empathy building is a common stage of design thinking and human centered design research in which researchers ``set aside their own assumptions'' to get to know user's real needs \cite{plattnerDesignThinkingUnderstand2011,wrightEmpathyExperienceHCI2008}. This stage is used to uncover research questions or design problems. Merriam Webster defines empathy as ``the action of understanding, being aware of, being sensitive to, and vicariously experiencing the feelings, thoughts, and experience of another without having the feelings, thoughts, and experience fully communicated in an objectively explicit manner'' \cite{inc.MeriamWebsterDictionaryDefinition2004}.  Empathy has long been considered an important component of participatory design as well as research processes. %Through the empathizing process, designers and researchers spend time getting to know their users and understanding their needs, wants, and objectives.

When used correctly, we believe empathy building can be a powerful first step before beginning participatory design with PwD. A common approach to empathy building is to imagine yourself in someone else's shoes (ie. what would it be like if I was 10 feet tall?). This is not appropriate to this context. Researchers and designers working with PwD should be careful to use empathy building to better understand ``being with'' rather than ``being like'' or ``vicariously living through'' users as the definition suggests \cite{bennettPromiseEmpathyDesign2019}. To be clear, it is not appropriate or effective to do empathy building for PwD though ``trying on'' their disabilities \cite{abreuWhyWonTry2018}.  Immersing oneself in the content created by PwD is a way to begin understanding communities before engaging directly with community members. Taking on this investigation of the community before (but not as a substitute for) working directly with users can be useful for building a basis for appropriate future interactions. 

This technique could be one possible step toward alleviating ``the potential over reliance and under acknowledged use of people with disabilities for their `access labor' in participatory design'' \cite{mackWhatWeMean2021,bennettBiographicalPrototypesReimagining2019}.
We encourage researchers to take on some of this initial labor by engaging with and celebrating the pre-existing cultural labor of PwD \cite{piepzna-samarasinhaCareWorkDreaming2018}. Researchers can build authentic understandings of the needs of populations with disabilities as a precursor to participatory work without putting undue access work on the community.  

This paper examines how close readings of media produced by PwD can lead to productive empathy building and the discovery of authentic, meaningful research agendas. We propose a method for building this empathy called DREEMing (\underline{D}isability-\underline{R}elated \underline{E}mpathy from \underline{E}xisting \underline{M}edia), which builds on other influential methods from various disciplines described in the Background section \ref{Background}, contemporary critiques of the SIGACCESS field discussed in the Motivation section \ref{Motivations}, and our own experience of empathy building through {\color{red}three} case studies presented in the Case Studies section \ref{CaseStudies}. DREEM is a 5-step process that we describe in Section \ref{DREEM}—with actionable tips and insights for how to find appropriate content and develop authentic research agendas. Finally, we discuss DREEMing's novelty, relevance, recommendations for working with research assistants, challenges, limitations, and future work.


\section{Background} \label{Background}
 In this section, we begin by discussing the motivations behind DREEM, which include sharing labor with people with disabilities, technosolutionism, contemporary critiques posed to the SIGACCESS community from Disability studies scholars, and documented challenges of employing co-creative, participatory methods with people with disabilities. Next, we discussed the methods and approaches we are inspired by that influenced DREEM. 

\subsection{Motivations}\label{Motivations}

A recent survey of ASSETS and CHI accessibility work showed that only 16 methodological contributions (3.2\% of all)  have been made to the accessibility community since 1994  \cite{mackWhatWeMean2021}.
\textbf{What does it mean to create a method that values disabled contributions? } Our guiding principles toward this value are:
\begin{itemize}
\item Honor the work that is already done and value contributor's time. \textbf{Labor \ref{Labor}}
\item Don't come to the table with a solution already in mind. \textbf{Technosolutionism \ref{Technosoultionism}}
\item PwD are not medical patients. \textbf{Authenticity \ref{authenticity}}
\item Make disabled contributions meaningful and accessible. Give proper acknowledgment and compensation. \textbf{Co-Design \ref{cocreationchallenges}}
\end{itemize}


\subsubsection{Labor} \label{Labor}

We acknowledge the labor required by people with disabilities in both designing and living with existing assistive and accessible technologies. In designing methods that center contributors with disabilities, labor should be carefully considered. Access labor refers to the work that people with disabilities are required to do in order to have their access needs met \cite{bennettBiographicalPrototypesReimagining2019,mackWhatWeMean2021,piepzna-samarasinhaCareWorkDreaming2018} . This can mean maintaining friendly relationships with caregivers \cite{kittayLoveLaborEssays1999}, requesting specific accommodations from event leaders, or the everyday work of living with a disability in an ableist world \cite{hanssonEthicsEnablingTechnology2007a}. 

There's also often a labor cost to using technology even after it has been developed. For example, Forlano, a professor of design who sometimes writes about her experience with type one diabetes, must frequently remove herself from meetings and gets woken up in order to recalibrate her automatic insulin pump \cite{forlanoDangerIntimateAlgorithms2019}. Weise is a poet, artist, and author who speaks of the labor needed to use a prosthesis, dealing with insurance companies, and walking just enough (but not too much) for them to warrant providing a leg \cite{weiseCommonCyborg2018}. 

Cultural labor is the organizing and creative work done to contribute to a particular culture such as disability cultures \cite{charltonNothingUsUs2000}. Cultural labor can be in many forms of advocacy including books (e.g., Nothing About Us Without Us \cite{charltonNothingUsUs2000}), legislation (e.g., the Disability Act \cite{civilrightsdivisionAmericansDisabilitiesAct}), or shared accounts (e.g., Resistance and Hope \cite{wongResistanceHopeEssays2018}).  Existing cultural labor is what DREEM relies on. 

\subsubsection{Technosolutionism} \label{Technosoultionism}
Too often, able-bodied scholars wave their techno-magical wands to try and fix problems they believe people with disabilities face \cite{morozovEverythingClickHere2013,charltonNothingUsUs2000}. A prime example of this phenomenon was beautifully articulated by Karen Nakumura during the 2019 ASSETS keynote where Nakumura argued, among other examples, that people who are blind do not want smart white canes because technology dies inconveniently, needs regular charging, is heavy, expensive, and can become a spectacle \cite{nakamuraMyAlgorithmsHave2019}. The white cane is already well designed, so why fix something that already works when there is much real work to be done? Nakumura also posited that if the engineers had simply asked a blind person whether or not they were interested in a smart white cane, they would have quickly moved on.

\subsubsection{Authenticity} \label{authenticity}
Over a decade ago, ASSETS scholars called for the use of a critical disability lens while designing and developing assistive technology for disabled individuals \cite{mankoffDisabilityStudiesSource2010}. This call has only strengthened in the proceeding years, with an emphasis on allowing for more co-design and co-research with disabled people \cite{bennettPromiseEmpathyDesign2019,ymousAmJustTerrified2020,spielAgencyAutisticChildren2019}. To summarize the concern, the majority of assistive technology devices and applications are rooted in medical discourse. That is, disability is an inherent problem in the body and must be ``fixed'' or ``normalized'' by intervention. Using a more socially-oriented lens, such as those found in disability studies, emphasizes the social context and environment as \textit{creating} disability by denying access to particular body configurations \cite{ginsburgDisabilityWorlds2013,ringlandPlacePlayDis2019,titchkoskyQuestionAccessDisability2011}. Given this concern about the discourses that influence the design of assistive technologies, it has become even more important for researchers to acknowledge the needs of the disabled individuals the technology is meant to help. DREEMing can potentially offer a lens into the sociocultural fabrics of disabled communities.

\subsubsection{Co-Creation and Participatory Design Challenges with PwD} \label{cocreationchallenges}
There are many successful examples of co-creating with PwD (e.g., \cite{anthonyParticipatoryDesignWorkshop2012,assadMotionBasedGamesParkinson2011,bentonDevelopingIDEASSupporting2012,duvalSpokeItBuildingMobile2018,ellisIncreasingUsabilityOnline2000,ellisParticipatoryDesignInternetbased1998,elorImmersiveVirtualReality2018,karnaDesigningTechnologiesChildren2010,gerlingDesigningMovementbasedPlay2016,grafIGYMInteractiveFloor2019,khaledBridgingSeriousGames2014,malinverniParticipatoryDesignStrategies2014,priorHCIMethodsIncluding2010}), which we discuss in Section \ref{PD}, but there are also many documented and pragmatic challenges \cite{wardReflectionsParticipatoryAction2001}. The high rates of assistive technology abandonment are due, in part, to the inability to take populations’ perspectives into consideration \cite{gitlinWHYOLDERPEOPLE1995,phillipsPredictorsAssistiveTechnology1993,vandijkEmpoweringPeopleImpairments2016}. If user input has proven to be valuable to designs where the contributions of populations with disabilities are included \cite{karnaDesigningTechnologiesChildren2010,sampleBeginningsParticipatoryAction1996}, why then are many systems designed without leveraging it? Documented obstacles to implementing participatory design include identifying users, obtaining access to users, motivating co-designers, ineffective feedback strategies, time constraints, issues identifying consensus, limited design competencies, overwhelming amounts of data, difficult scheduling logistics, and costs of developing changes \cite{kujalaUserInvolvementReview2003}. We believe many of these challenges can be alleviated with more preparation and empathy—often our assumptions and inaccessible protocols are the root of the issue, not disability. Kujala suggests that these challenges stem from arriving to the table with a prototype before doing appropriate ground work \cite{kujalaUserInvolvementReview2003}, similar to our technosolutionism arguments above.

Co-Design sessions should be valuable to all parties \cite{bodkerParticipatoryDesignThat2018}, especially as they disrupt everyday life and require participants to invest their precious time.  `Informant fatigue' happens when PwD (usually those with visible disabilities) are too often asked to share details about their disability with others who ask \cite{shinoharaSelfConsciousSelfConfidentDiary2016}. This can also take the form of `forced intimacy', where a PwD has to divulge what would normally be highly personal information in order to have their access needs met in a given context \cite{mingusForcedIntimacyAbleist2017}. Both informant fatigue and forced intimacy can occur in design research with PwD if researchers are not careful. Work remains to be done in centering PwD as designers for the contributions that they make \cite{bennettBiographicalPrototypesReimagining2019} and for compensating co-designers appropriately. We posit that DREEMing ahead of participatory sessions can alleviate the need to use precious time for learning the basics about a population and instead focus on actual co-design work -- giving them meaningful ownership as co-designers rather than collecting overly personal information.
			
\subsection{Influential Methods and Approaches}\label{influences}
As mentioned in Section \ref{Motivations}, there are relatively few methodologies designed specifically for use by the SIGACCESS community. Accessibility research draws on existing research methods such as those from human computer interaction, computer science, design, psychology, and sociology. In this section we discuss the multidisciplinary methodologies that informed the creation of DREEM. 

\subsubsection{Close Readings\label{CloseReadings}}
Close readings are the careful, deliberate observation of an artifact \cite{brummettTechniquesCloseReading2019}. Close reading is about \textit{seeing} what is there (and not there). It is about mindfulness, noticing, and reflection \cite{brummettTechniquesCloseReading2019}. Close readings afford the wandering mind to ask questions about what is and is not present and reflect on the possibilities in a larger context. Close readings typically are conducted on text, but they can also be applied to other designed artifacts such as games (e.g., \cite{wardrip-fruinHowPacManEats2020}), software (e.g., \cite{sackSoftwareArts2019}), videos {\color{red}[ref film?]}, and images {\color{red}[ref political cartoon?]}. Close readings are typically conducted by scholars in the humanities, but there is potential for designers to leverage close readings more broadly. If a close reading is the mindful, disciplined reading of an object with a view to deeper understanding of its meanings \cite{brummettTechniquesCloseReading2019}, then it has the potential to help us understand the experience of living with a disability more deeply. In short, close readings can help us build empathy. DREEMing relies on close readings and provides suggestions on how to find relevant media.

\subsubsection{Netnography}
Netnography is an online research method originating in ethnography and is often employed by social scientists and anthropologists \cite{kozinetsNetnographyEssentialGuide2019}. Instead of focusing on typical embodied phenomena in ethnography such as body language, netnography focuses primarily on the context of online media such as text and multimedia \cite{bartlReviewAnalysisLiterature2016}. Netnography is typically conducted on a smaller scale than sentiment analysis run on large data sets and provides more nuanced behavioral findings than automated software. Since Netnography uses spontaneous data and conducts observation without intruding online users, it is regarded as more naturalistic than other approaches such as interviews, focus groups, surveys and experiments \cite{kozinetsNetnographyRedefined2015}. These online community members often share in-depth insights on themselves, their lifestyles, and the reasons behind the choices they make \cite{kozinetsNetnographyEssentialGuide2019}.  For DREEM, these communities are populations with disabilities and the media discovered provides a basis to conduct a close reading for empathy building.

\subsubsection{Situated Play Design}
Situated Play Design (SPD) is an open methodological framework for surfacing existing manifestations of play in everyday life to inspire technology design \cite{altarribabertranChasingPlayPotentials2019a,altarribabertranDesigningPlayThat2019}. SPD has been used on TikTok to find playful content from creators with disabilities to inspire playful everyday technology (not necessarily assistive technology) \cite{duvalChasingPlayTikTok2021}. The 3-step approach outlined in SPD served as a model for us in finding online content for contextual technology design. DREEM differs from SPD in that we are not focused on play in this work and instead of directly inspiring technology design, DREEM is focused on developing research agendas. DREEM has the same roots in Research through Design \cite{gaverWhatShouldWe2012,zimmermanResearchDesignMethod2007} as SPD.

\subsubsection{User-centered Design and Participatory Design} \label{PD}
From exergames for wheelchair users \cite{gerlingDesigningMovementbasedPlay2016} to speech therapy \cite{duvalSpokeItCoCreatedSpeech2018}, virtual reality for teaching people with developmental disabilities to identify emotions in others  \cite{thangPhDForumStrengthening2018}, and robots for physical rehabilitation \cite{marquezseguraPlayificationPhySeEarCase2016}, technology can effectively be designed with and for people with disabilities. For facing the next generation of big issues that matter, all stakeholders should participate in the design of technology they will use \cite{bodkerParticipatoryDesignThat2018}. DREEM fits within the larger umbrella of participatory methods by leveraging existing cultural work to educate researchers prior to co-design sessions so that they can be more effective and appropriate. DREEM is not a replacement for participatory work, it is a precursor.

\subsubsection{Autoethnography}
``Autoethnography is a theoretical, methodological, and (primarily) textual approach that seeks to experience, reflect on, and represent through evocation the relationship among self and culture, individual and collective experience, and identity politics and appeals for social justice. In investigating these relationships, autoethnography fuses personal narrative and sociocultural exploration. Autoethnographic inquiry and writing has long been practiced by journalists and novelists, historians and biographers, travelers and journal writers'' \cite{holmanjonesAutoethnography2007}. ``Autoethnography refers to both a research process and the product of the approach. Practitioners draw from their lived experiences as a starting point for social inquiry. They represent their thoughts, emotions, collective experiences, and social processes associated with an identity or issue and then contextualize them in broader, societal‐level phenomena'' \cite{ramboAutoethnography2020}. This approach may employ a standard written essay format, a diary log, or handwritten annotations as well as more artful forms such as plays, art, music, and poetry. Autoethnography is an appropriate reflective practice while conducting close readings to develop empathy. DREEM pairs close readings with autoethnography to generate the appropriate type of empathy that can make participatory work more productive and appropriate.

\subsubsection{Inductive Thematic Coding}
Thematic analysis allows researchers to explore themes (overarching categories of common data) with the aim of  understanding emerging phenomena that appears and communicate findings with other researchers \cite{guestAppliedThematicAnalysis2012}. Inductive coding is especially relevant for the creation of new research agendas because it allows us to find novel areas, problems, and gaps to focus on. Inductive thematic coding is relevant to DREEM because it offers a grounded and established basis for generating research questions and communicating findings. 


\section{Method}
Our research questions that inspired the development of DREEM and this paper are: 
\begin{enumerate}[label=RQ\arabic*:]
\item \textit{How can the SIGACCESS community develop research agendas that are genuine to the needs of people with disabilities through appropriate empathy building?}
\item \textit{What insights does DREEM offer through employing it in case studies?}
\item \textit{How can scholars adopt DREEMing efficiently?}
\end{enumerate}

For \textit{RQ1}, we conducted a comprehensive literature review presented in the background section that includes motivations and influential works that drove the development of DREEMing. \textit{RQ2} is related to the \textit{Value} of DREEM, which we illustrate through our case studies. We use this section to describe how we conducted DREEMing with team of researchers towards iterating on the method itself and building the case studies. For \textit{RQ3}, we use autoethnographic reflections presented in the discussion section to share the insights, limitations, and value of DREEMing. 

To evaluate the effectiveness of DREEMing, develop tools to assist with its implementation, and develop insights on its future and limitations (\textit{RQ2}), we employed the method while iterating on its implementation through {\color{red}three} case studies which we present in the following section. In this section, we describe the process of employing DREEM with {\color{red}X} research assistants.

Prior to recruiting undergraduate research assistants, the senior research team, who designed DREEM, completed steps 2 and 3 on\href{https://www.youtube.com/watch?v=B1sWtT-wShI}{The Power of Choice} independently. Afterward, we collaborated on tweaks to the data collection and included our findings as a case study in the training materials we developed. We recruited {\color{red}X} RAs through department news letters and by advertising in classes we teach. Our research flier is included in an editable form in our supplementary materials. We accepted all applications. We then used When2Meet to find a time everyone was available for a on-boarding training session. The hour-long training described the motivations of the work, instructions on how to carry out the work, and expectations. An editable version of our training slides are available in the supplementary materials. We asked undergraduates to reflect on their interest and confirm whether or not they wanted to participate as a collaborator. RAs then used our DREEM form (available in the supplementary materials as an editable google form) to independently and asynchronously conduct steps 1-3. We had a reoccurring weekly check in where we  discussed progress, research directions, and reflections as a team. Our specific case  studies were born from exploration and interest-driven directions lead by the RAs. After 3 weeks, interns participated in the inductive data analysis and helped us construct the write-up for the case studies presented in this paper.

\subsection{Ethics}\label{Ethics}
A tricky element of our research is discovering existing content on social media and the ethical implications of researching on these platforms. Our data collection method closely aligns with \textit{Netnography} \cite{kozinetsNetnographyEssentialGuide2019}, which has established ethical guidelines \cite{kozinetsEthics2019}. These include the notion of public versus private information on social media, whether to anonymize or cite participants, and informed consent. Kozinets argues that ethical procedures must be decided on a case-by-case basis contingent upon the topic matter, the research purposes and the research approach of the particular \textit{netnography} \cite{kozinetsNetnographyEssentialGuide2019}. Some platforms such as Facebook and Instagram have varying levels of security and privacy settings for content and profiles that complicate what is truly public. Researchers using platforms with privacy settings must respect what is considered public and not. Bassett and Kozinets argue that when the internet is used as a ``megaphone-like'' public broadcasting medium \cite{mcquarrieMegaphoneEffectTaste2013}, we can thus perceive it as a form of cultural production, in a similar framework to that of the print media, broadcast television and radio where we should cite the source so that broadcasters can be credited for their work \cite{bassettEthicsInternetResearch2002,kozinetsEthics2019}. Sometimes content posted online is ephemeral such as temporarily available stories. To respect creators wishes, we did not include these media in our case studies. Because this research is minimal risk and fits the notion of public broadcasting online, we provided links to the original content in our published materials to respect the creators' (ongoing) decisions concerning public access to the videos. \textit{Netnography} of public archival content (not active research interventions such as interviewing) such as the media discovery methodology employed in this research would be unduly complicated with informed consent because the manual, non-automated access by researchers of public information should be acceptable without special permissions or actions \cite{allenAcademicDataCollection2006} and removing information from unreachable broadcasters would undermine researchers' ability to contribute to society \cite{kozinetsEthics2019}. Therefore, we included all applicable data scraped in our supplementary materials and make every effort to represent the content in this publication respectfully and in a positive light. Finally, our data was manually discovered without any automated system or software and is not used for commercial purposes, and therefore, at the time of writing to the best of our knowledge, adheres to the terms of services of these platforms.

\section{Case Studies} \label{CaseStudies}
To develop the DREEM Method we began with these 3 steps:
\begin{enumerate}
\item Discover Existing Media
\item Close Reading
\item Autoethnographic Reflections
\end{enumerate}
The rationale for these steps is to meet the criteria for finding existing sources of works created by PWDs (Discover Existing Media) and to analyze these works for understanding our disability space of interest (Close Reading). Finally, in order to be able to iterate on the design of this method, we included a step for a meta understanding of the method itself by reflecting on the process of the previous two steps (Autoethnographic Reflections). Including this third step allows us to continue refining and adding to this method, as we will detail later in the paper (Section \ref{DREEM}).

\subsection{Tourette Syndrome Case Study}
\begin{itemize}
\item Here's what it looks like on tiktok, youtube 
\item Here's who is making them. 
\item Here's some thoughts we have about whether or not these findings extend
\end{itemize}


\subsubsection{Summary of Case}

\subsubsection{Visual Representation of Case}

\subsubsection{Autoethnographic Reflections}

\subsubsection{Findings}
SMART Goals

\subsection{Travel & Vision Impairments}
\begin{itemize}
\item What happens when multiple people review the same sources
\end{itemize}

\subsubsection{Summary of Case}

\subsubsection{Visual Representation of Case}

\subsubsection{Autoethnographic Reflections}

\subsubsection{Findings}
SMART Goals

\subsection{Mobility}
\subsubsection{Summary of Case}

\subsubsection{Visual Representation of Case}

\subsubsection{Autoethnographic Reflections}

\subsubsection{Findings}
SMART Goals

\subsection{Beauty products and fashion}
\subsubsection{Summary of Case}

\subsubsection{Visual Representation of Case}

\subsubsection{Autoethnographic Reflections}

\subsubsection{Findings}
SMART Goals

\section{DREEM} \label{DREEM}
In this section, we introduce the 5 resulting steps to DREEMing. The process of completing this method is inspired by the influences discussed in the previous sections. Each of the five steps are discussed in detail with tips and insights derived from employing the method in the case studies described in the following section (Section \ref{CaseStudies}). The nature of DREEMing is qualitative and, therefore, a quantitative evaluation of the method is not appropriate (at least until more case studies adopt the method, when a systematic analysis can be conducted). Instead, we illustrate the value of DREEMing through the contributions of {\color{red}three} case studies and through the autoethnographic reflections \cite{ramboAutoethnography2020} in our Discussion (Section \ref{Discussion}). Steps 1-3 can be repeated for as long as necessary until data saturation is reached. For data saturation, we recommend using a diverse range of platforms, media mediums, and finding numerous subgroups within the target community.  It is possible that steps 1-3 will need to be repeated based on the results of the verification stage. 

<Step 4 and 5 grew out of the process (Research through Design)>

\subsection{Step 1: Discovering Relevant Media}
We consider any {\color{red}public??} medium to be of potential for DREEMing including blogs, images, videos, films, tweets, and posts. So far, we have primarily considered {\color{red}videos and essays} in our case studies. Mediums such as visual art are certainly possible, but require further exploration and a grounding in visual studies.  We focus on media that can be found online for ease of access. There does, of course, exist important media made by PwD that cannot be found online. It is possible to close read in-person performances, but having a recorded version offers the ability to sit with and return to the content. Zines and event ephemera could also offer interesting insights. This method could reasonably be extended to any of the above (and more!).

\subsubsection{Tips for Success}
Finding media created by PwD online can be a surprisingly difficult task. When looking for content from creators with autism, searching for ``autistic'' might seem like a good place to start. Unfortunately, it will likely result in informational content such as the biology of a disability rather than the lived experience or perspectives from a non-disabled creator. Furthermore, finding such media may require some prior community knowledge (hashtags, vocabulary, etc.) that may be difficult to access for an outsider. We offer the following tips for success in finding content creators with disabilities online. Examples of each are listed with each tip. 
 
\begin{itemize}
  \item Search common content with flavor: `what's in my bag: chronic illness edition', `amputee morning routine'
  \item Learn community hashtags and keywords, which vary from platform to platform: \#ActuallyAutistic, \#Spoonie, \#CripTheVote, \#ADHDTwitter
  \item Train the curation algorithm: Create a new social media account and follow only creators with hearing impairments as you find them. 
  \item Snowball: Discover accounts that a creator you follow tags. 
  \item Look for collectives and anthologies: SinsInvalid, Disability Visibility Project
\end{itemize}

\subsubsection{Potential Pitfalls}
Media found may not perfectly represent the community as a whole. Here are several pitfalls to consider when finding media. Things to look out for include any media that perpetuates ableism, only exposing yourself to a small subgroup of a larger community, and anything that has been shared exclusively with a private network (See Section \ref{Ethics}).

\subsection{Step 2: Close Reading}
This step requires reading or observing the media, and sitting thoughtfully with it as described in Section \ref{CloseReadings}. We recommend working systematically and using standardized collection measures. We include an editable Google form for DREEMing in the supplemental materials. Relevant details to log beyond the close reading itself include the source of the media, a short 1-5 word summary that makes skimming the data later easier, annotated screenshot(s), location in the media the close reading entry relates to (e.g., line number, time span in video), and keywords/tags. We enter each ``complete thought'' as one unit—these could be a few words or a few sentences. We also logged questions we asked ourselves that arose during the close readings. You can choose to immediately start logging your visceral reactions or start entering close readings after you've been fully exposed to the media, but we recommend doing both.

\subsubsection{Tips for Success}
Record your thoughts as they occur. These can directly relate to the content in the video or can be personal to your lived experiences. As you go, maintain a list of keywords and tag each recorded thought. These keywords can make indexing easier later. Our team took advantage of google forms and spreadsheets for this step. In general, take your time through this step. It may be useful to step away from the media and come back. Multiple reads may lead you in different directions. 

\subsection{Step 3: Reflection and Empathy Building}
Reflection is a crucial part of DREEMing. The primary aim of DREEM is to learn about communities in an authentic and lasting manner. Reflection creates the time and space to absorb your learnings and connect them with one another. Reflection is an important part of making sustainable perspective change. \cite{lyubomirskyPursuingHappinessArchitecture2005} We recommend doing a session of autoethnigraphical journaling after each analyzed media artifact. 
Maintaining a paper trail of your evolving thoughts also allows you to incorporate the learning process itself into the content analyzed via inductive thematic coding. 

\subsubsection{Tips for Success}
We like using the following prompts for our reflections. You do not need to make each reflection similar in structure to one another, and can choose or combine prompts as they seem relevant. 
\begin{itemize}
\item What trends or patterns do you see emerging?
\item Have you learned anything new about the community you are studying?
\item What could you improve about your logging process?
\item What is valuable or not valuable to you as an individual about your process? 
\item If working with others, what are similarities and differences you are seeing in your logging or retrospective writings versus your peers?
\item Have you learned anything that could inspire technology design?
\item What questions will you explore next and why? 
\end{itemize}
We often would quickly answer all of these questions in one diary-style entry or go in-depth with just one of them. We kept these prompts at the top of our diary documents to inspire us.

\subsection{Step 4: Generation of Research Agenda}
The goal of DREEMing is to provide a pathway into being an advocate for a more inclusive and accessible society through partnerships with communities of PwDs. There is much work to be done and it can be overwhelming—it is important to plan and focus on a specific idea or subset of the field or else it will seem unmanageable. At some point you'll reach saturation from completing steps 1-3 iteratively and hopefully have some ideas. It is also important to prepare for the inevitable evolution of research and the new ideas that will naturally emerge. 

\subsubsection{Tips for Success}
Developing research agendas and research questions is an art in itself. We recommend writing the research paper you hope to publish prior to collecting any data or doing any design work—the process will inform the questions you ask and \textit{how} you ask them. When forming research questions it may be helpful to apply the SMART (Specific, measurable, attainable, reasonable, timely) model strategically \cite{fieldingTargetSettingPolicy1999}. Ask yourself:
\begin{itemize}
\item Is this a agenda true to the authentic experience of the community?
\item Am I the right person to approach this work? 
\item Is the scope possible to tackle?
\item Should technology be used in relation to this experience, or would a different intervention be more appropriate?
\item Does the agenda uplift and support the community at all stages?
\end{itemize}

\subsection{Step 5: Verification}
Verification looks different based on your goals and intentions. 
Choose your adventure (it depends on where you want to go from here: presenting long winded DREEM Findings or beginning design work (is your contribution a humanities or a design contribution)):
\begin{itemize}
\item Do PD and verify through interactions with target users
\item You now have insights—use them to find appropriate literature that supports your findings
\item Reflect on whether or not it is reasonable for the findings to be generalizable to the broader specific community
\item Do interviews if you want to present DREEM in isolation
\end{itemize}

Verification of the experiences of, and validation of the research agenda by the community in question are necessary to this process. DREEM is not meant to stand in place of working directly with PwD, but as a precursor to. It is altogether possible that the researcher drew the wrong conclusions or did not have a wide enough view of the community at large. Do not be discouraged if this is the case. DREEMing is an iterative process, and each cycle holds value.
Think about using other platforms, mediums, and sub groups within the population. 

Researchers should take the usual steps in conducting ethnographic research after using the DREEM method. It is recommended that researchers learn from populations as novices {\color{red}cite}. The knowledge acquired from DREEM should not override this. 

\subsubsection{Tips for Success}

In the case that your findings and agenda are not verified, establish next steps for the next cycle of DREEM. Some sources of inspiration leading back into step 1 include:

\begin{itemize}
\item Talk to representatives
\item Find literature
\item Find Statistics
\end{itemize}



\subsection{RESULTS}
here's a breakdown of what we looked at and an evaluation of all the team

-- surveey underegrads

\section{Discussion} \label{Discussion}
\subsection{Positionality}
We developed DREEM because we are members of the SIGACCESS community and want to design technology that better serves the needs of PwD. We are inspired by the conversations that took place at previous ASSETS conferences. We are driven to provide actionable insights in \textit{how} we as scholars and academics can do better. DREEM attempts to add a procedure behind these discussions. The team of researchers who developed DREEM work in the <anonymous> Lab focusing on the development of assistive technology at <anonymous> University in the Computational Media department, which is situated in an Engineering environment. The interdisciplinary Computational Media Department bridges engineering perspectives with the Arts and Humanities. We are familiar with social justice and critical disability studies, but are formally educated as engineers. Our team is made up of people with disabilities and people without disabilities, but each of us is engaged with a population of people with disabilities we are not a part of. We are scholars from varying points in our careers including undergraduates, PhD students, candidates, postdocs, and tenured faculty. We work with a myriad of populations including children with speech impairments, children with physical impairments, autistic people, people with visual impairments, stroke survivors, and people with developmental disabilities.

\subsection{How to Present DREEM Findings}
DREEM findings can be presented as their own findings, or as step within a larger body of work. In each case the presentation of the work will look slightly different. This method generates research questions, so it is likely that the findings will become a part of the larger body. 

In the case that DREEM is presented as the primary finding, the outcome may look similar to a traditional close reading that focuses on a particular topic and discusses multiple sources. \cite{cullenBetterWorldExamples} and \cite{mingusReflectingFridaKahlo2010} are two good, and very different, examples of close readings. 

If the findings are presented as a step in a research project{\color{red}\ldots maybe mention future work}

We encourage researchers to share important elements that are specific to the DREEM process such as links to all media analyzed (regardless of if it is discussed in the body of text), keywords and their frequency, and a thematic analysis of the individual close readings and/or reflections. 

\subsection{Novelty}
There has been much discourse on how the SIGACCESS community needs to do better in regards to technosolutionism, authenticity, and inclusive design, as we describe in the motivations section (\ref{Motivations}), but to our knowledge there has not been actionable processes towards addressing these issues. DREEM is a bridge between contemporary critiques and actionable steps towards building research agendas that will support a more inclusive and accessible society. DREEM is intended to support new scholars and those interested in contributing to the SIGACCESS field. 

DREEM has become increasingly relevant in the wake of COVID-19. It can be challenging to include people with disabilities for participatory work generally \cite{wardReflectionsParticipatoryAction2001}, but there is a specific added risk during a global pandemic. In addition to general guidelines that limit in-person contact, populations of people with disabilities often have medical needs that place them at higher risk from the COVID19 virus \cite{armitageCOVID19ResponseMust2020}. In light of quarantine, designers are employing creative methodologies to carry out remote design work that is usually done in situ \cite{whiteLearningCOVID19Design2020} (e.g., using games to educate the public about COVID19 and collect data \cite{lopezherna;ndezHealthcareGamificationSerious2020}). Many of these technologies are not accessible to people with disabilities \cite{annaswamyTelemedicineBarriersChallenges2020}. DREEM allows researchers to safely conduct preliminary research in preparation for participatory work when in-person protocols are safe after ubiquitous vaccination. 

DREEM surfaces and features existing work and labor of PwDs. The validity of research can be more rigorous if the source of inspiration is surfaced and credit is given where it is due. DREEM extends participatory design from being inclusive on \textit{how} something should be made to \textit{what} should be made in the first place.

\subsection{DREEMing as a Team}
DREEMing can be done on your own or as a team. If you plan to work as a team, we have several recommendations which we discuss in this section.

DREEMing can be an effective way for undergraduate research assistants and AT newcomers to become acquainted with PwD. DREEMing as a team offers the ability to discuss and build on one another's work. 

It is not required for multiple team members to do close readings of the same media. However, doing so can offer insights from multiple perspectives. 

Leave time in your research process to read each other's close readings and journals and meet to discuss them. Teams should work together to find a logging process that works for everyone. Expectations for quality and length of passages should be set and continually talked about. 


\subsection{Challenges \& Limitations}

This method requires access to content created and posted online. 

Content creators with particular disabilities are relatively few on some mediums due to accessibility issues. Memes and GIFs for instance are often posted without alt text,\cite{gleasonMakingGIFsAccessible2020} so the participation of screen reader users with Memes and GIFs may be lower. 

Different social media platforms have their own affordances and norms. It may be worth investigating multiple platforms. 

Researchers must be careful not to over generalize as not every disabled individual will be represented by those who are creating content online (i.e., a disabled individual who does not have access to creating certain media or has no desire to create content on social media may have a very different experience than someone who does have access and a desire to create and post content.) The verification step can help to neutralize this issue, but requires that you verify with a different subset of the population, which can be tricky to know. 


\section{Future Work}
In this paper, we present several case studies that demonstrate DREEM. However, the research agenda created by the findings of each case study have yet to be enacted. Application of this method to longer term projects is needed. The research team intends to follow up on the case studies presented here as their own research projects. 

Our work with DREEM so far primarily considers videos and text passages. More work should be done to DREEM with visual art and audio. 

We are excited to see what others come up with when using DREEM. 

\section{Conclusion}
%TODO: mirror RQs
In this paper, we propose a 5-step method for using close readings of media posted by people with disabilities to build empathy and authentic research agendas prior to participatory work. 

We explore three research questions: {\color{red}...copy and discuss}

The potential benefits of continuing this line of work include shared labor, authentic research problems, increased visibility of disability communities, and healthier partnerships with communities of people with disabilities. 

\balance
%\begin{acks}

%\end{acks}