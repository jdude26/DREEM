%\textit{Accessible} technology can be used by all people in the target audience regardless of disability status \cite{storyMaximizingUsabilityPrinciples1998}. Technology that is not accessible places a ``handicap'' \cite{badleyGenesisHandicapDefinition1995} on people who do not have equitable access to the services that technology is designed to provide \cite{hanssonEthicsEnablingTechnology2007}. A handicap is not a person's disability—it is the barriers that society and technology place on people with disabilities \cite{hanssonEthicsEnablingTechnology2007}. These barriers can be considered to impact an individual's ``fit'' in their environment \cite{garland-thomsonMisfitsFeministMaterialist2011}. Too often, making technology accessible is an act of retrofitting solutions to make the original tech usable by people with disabilities \cite{mottAccessibleDesignOpportunity2019}, when these populations should have been included in the first place, resulting in designs that are more accessible and user-friendly for everyone \cite{storyMaximizingUsabilityPrinciples1998}.  \textit{Assistive} technology is meant to serve a specific need of a population of people with disabilities and is meant to be used primarily by people with disabilities, caretakers, and medical professionals \cite{smithAssistiveTechnologyProducts2018}. Assistive technology should be accessible to the target population so that they can use it—and common co-occurring disabilities should also be considered \cite{turyginPrevalenceCooccurringDisorders2014}. 



===Text from Workshop Paper===

It can be challenging to include people with disabilities for participatory work generally \cite{wardReflectionsParticipatoryAction2001}, but there is a specific added risk during a global pandemic. In addition to general guidelines that limit in-person contact, people with disabilities often have medical needs that place them at higher risk from the COVID19 virus \cite{armitageCOVID19ResponseMust2020}. In light of quarantine, designers are employing creative methodologies to carry out remote design work that is usually done in situ \cite{whiteLearningCOVID19Design2020} (e.g., using games to educate the public about COVID19 and collect data \cite{lopezherna;ndezHealthcareGamificationSerious2020}). Many of these technologies are not accessible to people with disabilities \cite{annaswamyTelemedicineBarriersChallenges2020}. Co-Design sessions should be valuable to all parties \cite{bodkerParticipatoryDesignThat2018}, but they disrupt everyday life and require participants to invest their precious time. 

Too often, able-bodied scholars wave their techno-magical wands to try and fix problems they believe people with disabilities face \cite{morozovEverythingClickHere2013,charltonNothingUsUs2000}. The obvious solution is to include people with disabilities throughout the design process using participatory methods \cite{bodkerParticipatoryDesignThat2018,storyPrinciplesUniversalDesign2001}, but this requires labor from people with disabilities.  By employing close readings \cite{brummettTechniquesCloseReading2018} and netnography practices \cite{kozinetsNetnographyEssentialGuide2019} the myriad of content produced by people with disabilities, researchers can observe already-available research data to inform the design of their research agenda. After researchers complete this groundwork, they can verify their findings with the target populations and begin participatory work to solve authentic problems with people with disabilities.

Relying on small community sizes often means that few people take on a large portion of the access labor required to design new technologies using participatory design \cite{mackWhatWeMean2021,bennettBiographicalPrototypesReimagining2019}. We are interested in tapping into the tacit and contextual design knowledge of people with disabilities on platforms they are already using to share content for all of these reasons.  Social media is rife with potential design material. In some of our previous works \cite{duvalChasingPlayTikTok2021}, we chose to focus on TikTok because is an inherently playful social media platform and it is a safe way to engage with populations of people with disabilities during a pandemic in a way that is not disruptive to their everyday life. To capture play potentials \cite{altarribabertranChasingPlayPotentials2019} on TikTok we employed the Situated Play Design methodology \cite{altarribabertranChasingPlayPotentials2019a}. These play potentials inspired design concepts that could inspire future technology. A limitation of this work is the design concepts were not verified by people with disabilities and the populations that they might affect—a need that we hope to address in future work with the method we propose in this position paper. 


===Text from Workshop Paper===

We propose using close reading of content created by individuals with disability and posted to social media as a first step in gaining better empathy and understanding of a particular disability community. Close reading is ``the meaningful, disciplined reading of an object with a view to deeper understanding of its meanings'' \cite{brummettTechniquesCloseReading2018}. This method follows others in HCI who have examined various media to gain an understanding of respective users (e.g., \cite{alticeAmErrorNintendo2015,sackSoftwareArts2019}). Close reading aligns with methods found in ``netnography'' (i.e., collecting from the myriad of media made available by individuals across different social media platforms) \cite{kozinetsNetnographyEssentialGuide2019}.


===Text from Workshop Paper===

Content on social media is \textbf{abundant and readily available}. As the proliferation of social media via sites such as TikTok, YouTube, or Wordpress is increasing across various communities, we, as researchers, also seen an increase in access to communities that are otherwise difficult to access. This allows researchers and designers who would normally not have access to specific communities to immerse themselves and gain an understanding of these communities' culture and needs.

Utilizing existing content on social media as a first step to getting to know communities instead of going directly to community members encourages a better \textbf{division of labor} and less access labor for the community members. Furthermore, this method asks researchers to engage with the appreciation and uplifting of labor already done by people with disabilities. 

Surfacing existing media also highlights the importance of \textbf{visibility} of people with disabilities \cite{wongDisabilityVisibilityTwentyfirst2020,profitaWearItLoud2018,faucettVisibilityDisabilityAssistive2017,duvalChasingPlayTikTok2021}) both to people with disabilities and others. Researchers can take an active role in bolstering visibility by engaging with this media and making it a part of their research agenda. 

This method encourages people not to come to user centered design with a solution in mind (retrofitting) \cite{mottAccessibleDesignOpportunity2019}, but to develop their ideas as they get to know the community. We see this contribution as an opportunity for researchers to become rooted in the community via social media. Scrolling social media is often deemed unproductive and as such, not a part of the research agenda. We believe it has the potential to let researchers meet communities where they are already at. 
%People-centric vs technosolutionism \cite{morozovEverythingClickHere2013} retrofitting 

Close reading, as a method, comes from the humanities and using this method in HCI opens the door for more interdisciplinary \textbf{collaborations} with humanities researchers and scholars. We are interested in developing this method further so that it can be simple and approachable to a diverse range of scholars for easy adoption. 